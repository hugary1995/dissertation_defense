\section{The Variational Framework}

\sectioncover

\subsection{Thermodynamics}

\begin{frame}
  \vspace{-2em}
  \begin{columns}
    \begin{column}{0.4\textwidth}
      \begin{itemize}
        \item  Working with the Helmholtz free energy density $\psi$, variables we are concerned with are
              \begin{align*}
                \bs{\upphi}, \quad \defgrad^p, \quad \ep, \quad d, \quad T.
              \end{align*}
        \item Conservations and thermodynamic laws:
              \begin{align*}
                 & \dot{\rho_0} = 0,                                                                             \\
                 & \rho_0 \bta = \divergence \bfP + \rho_0\btb,                                                  \\
                 & \bfP\defgrad = \defgrad\bfP^T,                                                                \\
                 & f -\divergence \bs{\xi} = 0,                                                                  \\
                 & \dot{u} + \dot{k} = \mathcal{P}^\text{ext} + \rho_0 q - \divergence \bth,                     \\
                 & \dot{s}^\text{int} = \dot{s} - \dfrac{\rho_0 q}{T} + \divergence \dfrac{\bth}{T} \geqslant 0. 
              \end{align*}
      \end{itemize}
    \end{column}
    \pause
    \begin{column}{0.6\textwidth}
      \begin{itemize}
        \item Generalized forces are
              \begin{align*}
                 & \bfP = \bfP^\text{eq} + \bfP^\text{vis}, \quad \bfT = \bfT^\text{eq} + \bfT^\text{vis}, \quad Y = Y^\text{eq} + Y^\text{vis}, \\
                 & f = f^\text{eq} + f^\text{vis}, \quad \bs{\xi} = \bs{\xi}^\text{eq} + \bs{\xi}^\text{vis},                                    
              \end{align*}
        \item Following the Coleman-Noll procedure:
              \begin{align*}
                 & \bfP^\text{eq} = \psi_{,\defgrad}, \quad \bfT^\text{eq} = \psi_{,\defgrad^p}, \quad Y^\text{eq} = \psi_{,\ep}, \\
                 & f^\text{eq} = \psi_{,d}, \quad \bs{\xi}^\text{eq} = \psi_{,\grad d}, \quad -s = \psi_{,T}.                     
              \end{align*}
        \item Viscous forces follow from the dual kinetic potential $\Delta^*$:
              \begin{align*}
                 & \bfP^\text{vis} = \Delta^*_{,\dot{\defgrad}}, \quad \bfT^\text{vis} = \Delta^*_{,\dot{\defgrad}^p}, \quad Y^\text{vis} = \Delta^*_{,\epdot}, \\
                 & f^\text{vis} = \Delta^*_{,\dot{d}}, \quad \bs{\xi}^\text{vis} = \Delta^*_{,\grad\dot{d}}.                                                    
              \end{align*}
        \item To satisfy the second law:
              \begin{align*}
                \delta = \bfP^\text{vis}:\dot{\defgrad} + \bfT^\text{vis}:\dot{\defgrad}^p + Y^\text{vis}\epdot + f^\text{vis}\dot{d} + \bs{\xi}^\text{vis}\cdot\grad d \geqslant 0.
              \end{align*}
      \end{itemize}
    \end{column}
  \end{columns}
\end{frame}

\subsection{The variational statement}

\begin{frame}
  With $\mathcal{V} = \{ \defrate, \dot{\defgrad}^p, \epdot, \dot{d} \}$:
  \begin{block}{}
    \vspace{-0.5em}
    \begin{align*}
      \left( \mathcal{V}, \dot{s}, T \right) = \arg \left( \inf_{\mathcal{V}, \dot{s}} \sup_T L \right)
    \end{align*}
  \end{block}
  
  \begin{columns}
    \pause
    \begin{column}{0.6\textwidth}
      Benefits:
      \begin{itemize}
        \item From a theoretical standpoint:
              \begin{itemize}
                \item The direct method of \textcolor{peggyblue}{calculus of variations} informs conditions for the existence and uniqueness of solutions.
                \item Localization effects can be studied within the framework of \textcolor{peggyblue}{free-discontinuity problems}.
              \end{itemize}
        \item From a computational standpoint:
              \begin{itemize}
                \item Discretization leads to a \textcolor{peggyblue}{symmetric operator}.
                \item Discretization leads to robust and efficient variational constitutive update.
                \item The total potential can assist \textcolor{peggyblue}{line search}.
                \item The total potential can be directly used as an \textcolor{peggyblue}{error indicator} for adaptive mesh refinement.
                \item Many powerful optimization packages exist, e.g. PETSc/TAO, Trilinos, Matlab, etc..
              \end{itemize}
      \end{itemize}
    \end{column}
    \pause
    \begin{column}{0.4\textwidth}
      Limitations:
      \begin{itemize}
        \item Construction of such a potential is no easy task.
        \item (Other limitations will be discussed in the end.)
      \end{itemize}
      \pause
      \begin{block}{}
        \underline{State-of-the-art:} variational brittle fracture that concerns with
        \begin{align*}
          \bs{\upphi}, \quad d,
        \end{align*}
        while we are looking at
        \begin{align*}
          \bs{\upphi}, \quad \defgrad^p, \quad \ep, \quad d, \quad T.
        \end{align*}
      \end{block}
    \end{column}
  \end{columns}
\end{frame}

\begin{frame}
  \vspace{-1.5em}
  \begin{columns}[T]
    \begin{column}{0.6\textwidth}
      The total potential $L$ is constructed as
      \begin{align*}
        L       & = \int\limits_{\body_0} \varphi \diff{V} - \mathcal{P}^\text{ext}, \\
        \varphi & = \dot{k} + \dot{u} + \Delta^* - T\dot{s} - \chi,                  
      \end{align*}
      
      The external power expenditure $\mathcal{P}^\text{ext}(\defrate, T)$ is defined as
      \begin{equation*}
        \begin{aligned}
          \mathcal{P}^\text{ext} = & \ \underbrace{\int\limits_{\body_0} \rho_0 \btb \cdot \defrate \diff{V}}_\text{body force} + \underbrace{\int\limits_{\partial_t\body_0} \btt \cdot \defrate \diff{A}}_\text{surface traction} + \underbrace{\int\limits_{\partial_h\body_0} \bar{h}_n\ln\left( \dfrac{T}{T_0} \right) \diff{A}}_\text{external heat flux} \\
                                   & + \underbrace{\int\limits_{\partial_r\body_0} h\left[ T - T_0 \ln\left( \dfrac{T}{T_0} \right) \right] \diff{A}}_\text{external heat convection} - \underbrace{\int\limits_{\body_0} \rho_0 q \ln\left( \dfrac{T}{T_0} \right) \diff{V}}_\text{heat source},                                                               
        \end{aligned}
      \end{equation*}
    \end{column}
    \pause
    \begin{column}{0.4\textwidth}
      \begin{itemize}
        \item Mass balance and angular momentum balance are satisfied by construction.
        \item Linear momentum balance:
              \begin{align*}
                \rho_0\bta = \divergence\bfP + \btb.
              \end{align*}
        \item Micro-macro force balance:
              \begin{align*}
                \divergence \bs{\xi} - f = 0.
              \end{align*}
        \item Plastic flow:
              \begin{align*}
                \left( \dot{\defgrad}^p, \epdot \right) = & \ \arg\inf_{\dot{\defgrad}^p, \epdot} \left[ \bfT^\text{eq}:\dot{\defgrad}^p + Y^\text{eq}\epdot + \Delta^* \right] \\
                                                          & \ \text{subject to }\quad \bfL(\bfZ)\dot{\bfZ} = \bs{0}.                                                            
              \end{align*}
        \item Heat transfer:
              \begin{align*}
                T\dot{s} = \rho_0 q - \divergence \bth + \delta.
              \end{align*}
        \item (Strict) dissipation inequality requires $\Delta^*$ to be convex in each rate.
      \end{itemize}
      
      
    \end{column}
  \end{columns}
\end{frame}
